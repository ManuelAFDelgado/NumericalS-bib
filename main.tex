\documentclass[11pt]{amsart}
%\documentclass[11pt]{article}
%\documentclass{eptcs}
\usepackage{fullpage} 
\usepackage[T1]{fontenc}
\usepackage[utf8]{inputenc}

%%%%%%%%%%%%%%%%%%%%%%%%%% ``commonly used packages`` %%%%%%%%%%%%%%%%%%
\usepackage{amsfonts,amsmath,amsthm,amssymb}
\usepackage{enumitem} 
%%%%%%%%%%%%%%%%%%% end of ``commonly used packages`` %%%%%%%%%%%%%%%%%%

\usepackage[hidelinks,linktoc=all]{hyperref}

%%%%%%%%%%%%%%%%%%%%%%%%%%%%%%%%% "for references" %%%%%%%%%%%%%%%%%%%%%%%%%%%%%%%%%
% to use bibtex, comment the following lines and adapt the end of the document
%%%%%%%%%
\usepackage[backend=biber,style=numeric,hyperref=true,isbn=false,doi=false]{biblatex}
\addbibresource{numericals.bib}
\addbibresource{preprints.bib}
\addbibresource{refs.bib}
 %%%%%%%%%%%%%%%%%%%%%%%%%% end of "for references" %%%%%%%%%%%%%%%%%%%%%%%%%%%%

%%%%%%%%%%%%%%%%%%%%%%%%%%%% ``theorems`` %%%%%%%%%%%%%%%%%%%%%%%%%
%\newtheorem{theorem}{Theorem}[section] %(uncomment to number per section)
\newtheorem{theorem}{Theorem}
\newtheorem{lemma}[theorem]{Lemma}
\newtheorem{corollary}[theorem]{Corollary}
\newtheorem{proposition}[theorem]{Proposition}
%\newtheorem{algorithm}[theorem]{Algorithm}
\newtheorem{conjecture}[theorem]{Conjecture}
\newtheorem{fact}[theorem]{Fact}

\newtheorem{defn}[theorem]{Definition}

\theoremstyle{remark} 
\newtheorem{example}[theorem]{Example}
\newtheorem{remark}[theorem]{Remark}
\newtheorem{question}[theorem]{Question}
\newtheorem{problem}[theorem]{Problem}
\newtheorem{definition}[theorem]{Definition}
%%
\newtheorem*{remark*}{Remark}
\newtheorem{claim}{Claim}
\newtheorem{code}[theorem]{GAP-code}

%%%%%%%%%%%%%%%%%%%%%%%%%%%% end of ``theorems`` %%%%%%%%%%%%%%%%%%%

\title{Some references on numerical semigroups}
\author{Manuel Delgado
\email{mdelgado@fc.up.pt}
}
\address{CMUP--Centro de Matemática da Universidade do Porto,\\
Departamento de Matemática, Faculdade de Ciências,\\
Universidade do Porto,\\
Rua do Campo Alegre s/n, 4169– 007 Porto, Portugal}
\thanks{The first author was partially supported by CMUP, a member of LASI, which is financed by national funds through FCT – Fundação
  para a Ciência e a Tecnologia, I.P., under the projects with reference UIDB/00144/2020 and UIDP/00144/2020. He also acknowledges the Proyecto de Excelencia de la Junta de Andalucía (ProyExcel 00868).}
\date{\today}

\begin{document}
\keywords{Numerical semigroup, Kunz languages, Chomsky hierarchy}
\subjclass[2010]{20M14, 68Q45}
% 20M14 Commutative semigroups
% 20--02 Group theory and generalizations -- survey articles
% 05--02 Combinatorics -- survey articles
% 11--02 Number theory -- survey articles
% 20--04 Explicit machine computation and programs (not the theory of computation or programming) 
% 05A Enumerative combinatorics [For enumeration in graph theory, see 05C30] 
% 05A15 Exact enumeration problems, generating functions [See also 33Cxx, 33Dxx]
% 68Q45 Formal languages and automata [See also 03D05, 68Q70, 94A45]
% 68Q04 Classical models of computation (Turing machines, etc.) [See also 03D10]

\begin{abstract}
This document is associated to my bibtex database on numerical semigroups. 
\end{abstract}
\maketitle

\section{Books}\label{sec:books}
bla \cite{RosalesGarcia2009Book-Numerical}
\cite{Ramirez-Alfonsin2005Book-Diophantine}
\cite{Bras-AmorosRodriguez2021inproc-New}
%%
\section{Published papers}\label{sec:publshed-papers}
\cite{Eliahou2018JEMS-Wilfs}
%%
\section{Kunz languages}\label{sec:preprints}
\cite{Delgado2019ae-Trimming} 
%%%%%%%%%%% References %%%%%%%%
% in order to use bibtex comment the following line and uncomment the others
%% uncomment to use biblatex  
\printbibliography
%%
%%uncomment to use bibtex
%\bibliographystyle{plainurl}
%\bibliography{numericals.bib}
%%%%%%%%%
%\input{refs}
%%%%%%%%%%%%%%%%%%%%%%%%%%%%%%%%%%%%%%%%%%%%%%%%%%%%%%%%%%%%%%%%%%%%%
\end{document}
